\documentclass{amsart}
\title{Attack on some Galois RLWE instances}
\author{Hao Chen, Kristin Lauter, Kate Stange}

\usepackage{macros}


\begin{document}
\maketitle

\section{Introduction}

We will use an example to illustrate our attack. The main
difference from the [ELOS] attack will be
\begin{itemize}
\item We use the information on the entire singular value
decomposition of the embedding matrix (called $M_\alpha^{-1}$ in [ELOS]), where the authors considered the largest singular value only.

\item Our instances are Galois extensions of the rationals, making the search-to-decision reduction in [LPR] work for these instances.

\item We will not require the number field to be monogenic;
to compensate for a lack of power basis, we use algebra softwares such as sage to compute integral bases, and in the cases where a large portion of the basis vectors have no constant term, we can do a ``root zero'' attack, in contrast to the ``root 1'' attack in [ELOS].
\end{itemize}

\section{An example of degree $n = 20$}

Here we take a galois extension $K/\bQ$ of degree 20 defined by the polynomial

$$f(x) = x^{20} - 396 \, x^{15} + 300806 \, x^{10} - 63776196 \, x^{5} + 25937424601.$$
Note that $q = f(0)$ is a prime of fairly large size.

Let $a$ denote a root of $f$ in $K$. Then an integral basis $B$ of $K$ is computed by sage. We show some of the entries:

\begin{align*}
\cdots \\
\frac{254861}{1169230260} a^{18} + \frac{16}{8857805} a^{13} + \frac{1}{878460} a^{8} \\
\frac{1175891}{12861532860} a^{19} + \frac{527}{194871710} a^{14} + \frac{1}{9663060} a^{9} \\
\frac{43}{17715610} a^{19} + \frac{2247}{8857805} a^{18} + \frac{821}{1610510} a^{15} + \frac{3}{1610510} a^{14} + \frac{2}{805255} a^{13} + \frac{1}{146410} a^{10} \\
\frac{43}{17715610} a^{19} + \frac{4451}{8857805} a^{18} + \frac{821}{1610510} a^{16} + \frac{3}{1610510} a^{14} + \frac{1}{805255} a^{13} + \frac{1}{146410} a^{11} \\
\frac{4451}{17715610} a^{19} + \frac{4451}{8857805} a^{18} + \frac{821}{1610510} a^{17} + \frac{1}{1610510} a^{14} + \frac{1}{805255} a^{13} + \frac{1}{146410} a^{12} \\
\frac{1789}{3543122} a^{18} + \frac{1}{322102} a^{13} \\
\frac{1789}{3543122} a^{19} + \frac{1}{322102} a^{14} \\
\frac{1}{1331} a^{15} \\
\frac{1}{1331} a^{16} \\
\frac{1}{1331} a^{17} \\
\frac{1}{1331} a^{18} \\
\frac{1}{1331} a^{19} \\
\end{align*}

When we plug in $x = 0$, the basis $B$ becomes
\[
  \bar{B} =  [1/600, 0 \cdots, 0].
\]
This means if we choose $q = f(0)$ to be the modulus, then the only basis vector that survives the reduction to $\bF_q$
is the first one.

\section{Gaussian}

Let $M_B$ denote the embedding matrix of $B$, i.e., if
$B = b_1 ,\cdots b_n$ and $\sigma_1,\cdots, \sigma_{n/2}$ are
the complex embeddings of $K$ (Note that K has no real embeddings), then the i-th row of $M_B$ is
\[
Re(\sigma_1(b_i)), Re(\sigma_2(b_i)), \cdots, Re(\sigma_{n/2})(b_i),Im(\sigma_1(b_i)), Im(\sigma_2(b_i)), \cdots, Im(\sigma_{n/2})(b_i).
\]
We consider the singular value decomposition of the matrix $M_B^{-1}$:
\[
M_B^{-1} = U \Sigma V.
\]
where $U,V$ are real orthogonal matrices, and $\Sigma$ is a diagonal matrix with diagonal entries the singular values $\lambda_i$ of $M_B^{-1}$. Now taking an error vector amounts to multiplying $M_B^{-1}$ on the right with a column vector ${\bf x} = (x_i)$ sampled from a spherical Gaussian. Multiplying by an orthogonal matrix do not change that, so we may as well assume $V = Id$ for our analysis. Then the error vector we obtain is
\[
    e = \left(\sum_{i} u_{i1}\lambda_i x_i\right)b_1 + \cdots + \left(\sum_{i} u_{in}\lambda_i x_i\right)b_n.
\]
Let's denote ${\bf y_j} = (u_{ij} \lambda_i) \in \bR^n$. Then
\[
    e = ({\bf y_1} \cdot {\bf x}) b_1 + \cdots + ({\bf y_n} \cdot {\bf x}) b_n.
\]

We will use the form of this $e$ in the next section.

\section{The attack}

Note that when we take $x = 0$, all the basis vector except
$b_1$ are zero. So the error evaluated at zero will be
\[
    e(0) = [{\bf y_1} \cdot {\bf x}] \cdot \bar{b_1}
\]
where $[]$ means rounding to the nearest integer and
$\bar{b_1} = 1/600 \pmod{q} = $










\end{document}
