\documentclass{article}
\title{Title of Document}
\author{Name of Author}

\begin{document}
\maketitle

\section{Some observations}

For $p = 101$, modulus switching seems to fail at $d = 20$. It worked for other smaller degrees, and the smaller the degree is, the better the attack (it seems).

Maybe when the index is larger, it will be better?

Saturday: degree gets to 12, it is already uniform. (What?)

Maybe try some more general Galois extensions with BKZ. 



\section{Galois Split prime}

Galois instances vulnerable to the $\chi^2$ uniform test.


$p = 101, d = 10, q = 5437$.


\section{Modulus switching}

Instances vulnerable to modulus switching. Here
$r$ is the success rate, $d_v$ the adjustment factor, and
$\sigma$ the actual standard deviation used (ideally, $\sigma=  \sigma_0 d_v$).

$p = 101, d = 10, d_v = 8,  \sigma = 5, r = 63/100$.
Number of samples used is around 5000.

$p = 211, d = 14, d_v = 8.48, \sigma =4, r = 56/100$.
Number of samples: 4000.

$p = 307, d= 17, \sigma = 7, d_v = 14.8, norm bound = 13,  r= 40/100$

\end{document}
