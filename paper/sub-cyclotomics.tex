\documentclass{amsart}
\title{sub-cyclotomics}
\author{Hao Chen, Kristin Lauter \and Kate Stange}

\usepackage{algpseudocode}
%\usepackage{algorithmic}
\usepackage{algorithm}
\usepackage{macros}
%\renewcommand{\algorithmicrequire}{{\bf INPUT:}}
%\renewcommand{\algorithmicensure}{{\bf OUTPUT:}}

\begin{document}
\maketitle

\section{Introduction}

The fields considered in this section are subfields of cyclotomic fields $\bQ(\zeta_m)$, where we assume $m$ is {\it odd and squarefree}. The Galois group $Gal(\bQ(\zeta_m)/\bQ)$ is canonically isomorphic to $(\bZ/m\bZ)^*$.

{\bf Notation}: for each subgroup $H$ of $G = (\bZ/m\bZ)^*$, we use $K_{m,H}$ to denote the fixed field
\[
    K_{m,H} := \bQ(\zeta_m)^H.
\]

The extension $K_{m,H}/\bQ$ is Galois of degree $n = \frac{\varphi(m)}{|H|}$; a prime $q$ splits completely in $K_{m,H}$ if and only if $q \pmod{m} \in H$. In general, the degree of a prime $q$ in $K_{m,H}$ is equal to the order of $[q]$ in the quotient group $G/H$.


We search for vulnerable instances among fields of form $K_{m,H}$. The searching is done by generating actual RLWE samples from the instance and
run $\chi^2$ attack (Algorithm ~) on these samples. Success of the attack would indicate vulnerability.


The field searching requires sampling efficiently from a discete Gaussian $D_{\Lambda, \sigma}$. Hence one needs to compute an integral basis for $K$ and the embedding matrix $A_v$, which is time-consuming for general fields.  Luckily, every field of form $K_{m,H}$ always possess a {\it normal integral basis\}, which takes a simple form. In addition, its embedding matrix is easy to compute.


Let $K = K_{m,H}$ and let $C$ denote a set of coset representatives of the group $G/H$.

\begin{Definition}
For each $i \in C$, define
\[
    b_i =  \sum_{h \in H} \zeta_m^{hi}.
\]
\end{Definition}

\begin{Prop}
Suppose $m \geq 1$ is odd and squarefree. Then the elements
$(b_i)_{i \in C}$ form a $\bZ$-basis for $K_{m,H}$.
\end{Prop}

\begin{proof}
Application of Hilbert-Speiser theorem.
\end{proof}




To work with real matrices, following [DD], we define a matrix $T$
\begin{Definition}


\end{Definition}



Let  $n$, and let $\sigma_1, \cdots,\sigma_n$ be the embeddings of $K$ into the field of complex numbers. Assume that the $\sigma_i$ are ordered such that if $\sigma_i$ is a complex embedding, then $\sigma_{i+n/2} = \bar{\sigma_i}$.

\begin{Definition}
 For any sequence ${\bf a} = (a_1, \cdots, a_n)$  of $n$ elements in $K$, define the {\em canonical embedding matrix} of ${\bf a}$ to be
\[
    A_{\bf a}^0 = (\sigma_i(a_j))_{i,j}.
\]
Define the {\em real embedding matrix} of {\bf a} to be

$$A_{\bf a} =  \begin{cases} T^*A_{\bf a}^0 &\mbox{if } K \mbox{ is totally complex} \\
 A_{\bf a}^0 & \mbox{ otherwise}  \end{cases}$$
\end{Definition}

Note that the entries of $A_{\bf a}$ are always real numbers. In particular, if ${\bf a}$ consists of a $\bZ$-basis of $\cO_K$, then we could use the columns of $A_{\bf a}$ as the basis for our sampling purposes.


since by spherical symmetry and the property of the normal integral basis, the error distribution $D \pmod{\fq}$ is independent of the choice of $\fq$.


\section{Examples}

In table, we list some vulnerable instance we found. The columns are
as follows. Note that we ommited the prime ideal $\fq$ due to Lemma~.
$s = \sqrt{2 \pi} \sigma$ denotes the width of the error, and $t$ denotes the running time in seconds.

\begin{table}
\caption{Vulnerable sub-cyclotomic RLWE instances}
\begin{tabular}{c|c|c|c|c|c|c|c}
$m$ & gensH & $n$ & $q$ & $f$ & $\sigma$ & $M$ & $t$ \\ \hline
90321 & [90320, 18514, 43405] & 80 & 67 & 2 & 1 & 26934 & 17322.9 \\
\end{tabular}
\end{table}



\section{Proofs}


\iffalse
One might argue that in the $\chi^2$ attack, one needs to
guess $q^r$ values for $\bar{s} = s \pmod{\fq}$, and if run $\chi^2$
test on $q^r$ sets of independent uniform samples, it is conceivable that some of them will fail the test and these will mix up with correct guess, causing the attack to fail.
However, the samples from  ``wrong guesses'' of
$\bar{s}$ are not independent. In this section, we are going to show

First we set up some notations: if we have a list $T = (t_1, \cdots, t_N)$ of samples in $\bF_{q^r}$, let $f_T$ denote its relative frequency function:
\[
    f_T(a) := \frac{|\{ i : t_i = a\}|}{|T|}, \forall a \in \bF_{q^r}.
\]

\begin{Lemma} \hfill \\
(1) $f_T(0) = 1$, $||f_T||_1 = 1$, $||f_T||_{\infty} \leq 1$. \\
(2) Suppose $0 \neq s \in \bF_{q^r}$, then $f_{sT}(a) = f_{T}(a/s)$. \\
(3) Let the number of bins be $q^r$, then the $\chi^2$
statistics of $T$ is
$$\chi^2(T) = T ||f_T - u||_2 = ||\hat{f_T} - \delta||_2.$$
(4) Suppose $0 \neq s \in \bF_{q^r}$ and the number of bins is $q^r$. Then $\chi^2(sT) = \chi^2(T)$.
\end{Lemma}

\begin{proof}
Follows from definition. (4) is a consequence of (2) and (3).
\end{proof}

Let $(a_i,b_i), 1 \leq i \leq N$ be the samples generated.
Let $\bar{s}'$ denote our guess for $\bar{s}$. Then we obtained the ``errors''
\begin{align*}
    \bar{e_i'} &= \bar{b_i} - \bar{a_i}\bar{s}' \\
               &= \bar{a_i}(\bar{s}- \bar{s}') + \bar{e_i}.
\end{align*}

Let $A := \{\bar{a_i}\}$,  $E := \{\bar{e_i}\}$, $E' := \{\bar{e_i}\}$, and let $g = \bar{s}- \bar{s}'$. Hence we have
$f_E' = f_{Ag} \star f_{E}$. Taking Fourier tranform, we obtain $\hat{f_{E'}} = \hat{f_{A g}}  \hat{f_E}$.

Suppose $g \neq 0$, then we have
\begin{align*}
\chi^2(E')  & =  ||\hat{f_E} - \delta||_2 \\
           & = ||\hat{f_{gA}}  \hat{f_E} - \delta||_2 \\
          & = ||\hat{f_E}(\hat{f_{gA}}  - \delta)||_2 \qquad (\mbox{used the fact that } \hat{f_E} \delta = \delta) \\
          & \leq ||\hat{f_E}||_\infty \cdot \chi^2(gA) \\
          & \leq \chi^2(gA) = \chi^2(A).
\end{align*}
Hence we can actually improve our attack, right? Since we can compute $e\chi^2(A)$ exactly without making any guesses. No. These are all nonsense.

\begin{Prop}
Let $K = K_{m,H}$ and let $n_K = [K:\bQ]$. Let ${\bf b} = \{b_0, \cdots, b_{m-1}\}$. We have
$$A_{\bf b} A_{\bf b}^* = \sqrt{m|H|}I_{n_K}.$$
\end{Prop}


\begin{proof}
Straightforward computation.
\end{proof}

\begin{remark}
This proposition generalizes a part of Theorem 5 of [DD], which is the case when $|H| = 1$.
\end{remark}


\section{Scaling, and description for the attack}

Fix $m, H$. Let $K = K_{m,H}$. Let $q$ be a prime that splits completely in $K$. We are going to analyze the vulnerablity of $(K,q)$ to the evaluation attack.

Let $\fq_1, \cdots, \fq_n$ be the prime ideals in $\cO_K$
lying over $q$, and for each $1 \leq i \leq n$ let $\pi_i: \cO_K \to \cO_K/\fq_i \cong \bF_q$ be the natural projection. Luckily, it is sufficient to consider $\pi_1$, since the sequence ${\bf b} = (b_i)_i$ defined in () is $Gal(K/\bQ)$-invariant.

Let $\pi = \pi_1$ and we obtain $\bar{b_i} :=  \pi(b_i) \in \bF_q$ for $0 \leq i < m$. Let ${\bf \bar{b}} = (\bar{b_i})_i \in (\bF_q)^m$. Recall that the error term in our sampling approach is of form
\[
    e = \sum_{i = 0}^{m-1} e_i b_i,
\]
where $e_i$ are sampled independently from integer Gaussians with parameter $\sigma$ (the value of $\sigma$ is  determined in section 2). We apply $\pi$ and obtain

\[
    \pi(e) = \sum_{i=0}^{m-1} \pi(e_i) \bar{b_i}.
\]
We abuse notations and let $\bar{b_i}$ also denote its reduced representative in $\bZ$, i.e., the unique integer in $(-q/2, q/2)$ congruent to $\bar{b_i} \pmod{q}$. By the remark after Proposition~\ref{who knows?}, we know $\pi{e}$ is can be thought of as a discrete random variable with parameter
\[
    \sigma_\pi = ||{\bf \bar{b}}||_2 \cdot \sigma.
\]
We keep the assumption of [ELOS] that a discrete Gaussian
integer of parameter $\sigma$ has values in $[-2\sigma, 2\sigma]$. Now the error size will be $\max(2 \sigma_\pi, q)$ and the portion of elements of $\bF_q$ is
\[
    r(m,H) = \frac{2\sigma_\pi}{q} = \frac{2\sigma}{q} ||{\bf \bar{b}}||_2.
\]

\begin{theorem}
If $r = r(m,H) < 1$, the [ELOS] attack will work on $K = \bQ(\zeta_m)^H$. The smaller $r$ is, the more efficient the attack will be.
\end{theorem}

Note that [ELOS] required $r \leq 1/2$. It seems that in  practice, though, a value $r \leq 9/10$ is often good enough.


We recall a standard result from statistics.
\begin{Fact}
Suppose $x = (x_1, \cdots,x_n)$ is a random Gaussian vector in $\bR^n$ with covariance matrix $\Sigma$ and let $y = Ax$. Then $y$ is Gaussian with covariance matrix $A\Sigma A^T$.
\end{Fact}

A special case of the above fact is when $A = (a_1, \cdots, a_n)$ is a n by 1 matrix. In this case, let $s_{ij}$ be the i,j -th entry of $\Sigma$, then $y = \sum a_i x_i$ is a random varianble with variance

$$var(y) = \sum s_{ij} a_ia_j.$$

In particular, if $\Sigma = Diag(s_1, \cdots, s_n)$ is a diagonal matrix, then

$$var \left(\sum a_i x_i \right) = \sum s_i a_i^2,$$

an identity which we will later find useful.

\begin{theorem}
Let $y = (y_0, \cdots, y_{m-1}) \in \bC^m$ be a spherical Gaussian with parameter $\sigma$. Then the random vector
\[
   (\sigma_1(\sum y_ib_i), \cdots, \sigma_n(\sum y_ib_i)) \in \bC^n
\]
is a spherical Gaussian  with parameter $\sqrt{m|H|}\sigma$.
\end{theorem}

\fi


\subsection{Scaling}

The above analysis needs to be strengthened to take scaling into account. If $a \in \bZ$ is coprime to $q$, then the set of values of $ae$ and $e$ will have the same size, but this scaling multiplies the norm of the vector $||\bar{b}||_2$ by $a$. To deal with this issue, we considered scaling the vector ${\bf \bar{b}}$ by every $a \in \bF_q^*$ and find the one that yields the smallest 2-norm:
\[
    \sigma_{\pi, opt} = \min\{||a{\bf \bar{b}}||_2 : a \in \bF_q^*\}
\]
and
\[
    r_{opt} = \frac{2\sigma_{\pi,opt}}{q}.
\]


For examples, see these files:


\iffalse
\section{It's all about sampling}

Still looking at odd squarefree sub-cyclotomics. The thing is we have 3 different way of doing sampling:
first: we have GPV, second, we have coefficient-wise
discrete Gaussian (with a suitable $\sigma$ that we have to decide upon); third, we have [DD], which is: generate a continuous spherical gaussian in $\mathbb{R}[x]/(x^m-1)$, map down to $\mathbb{R}[x]/\varphi_m(x)$, and then round the coefficients to the nearest integers.

Remark: [DD, Theorem 2] is a hardness proof for this sampling method. This does not contradict our result since our sigmas are smaller than they have. They have
esesntially $\omega(\sqrt{m \log(m)})$, and it grants security; they mentioned the work of Arora and Ge that
states that if $\sigma$ is too small, then there would
already be a sub-exponential time attack.
\fi

\iffalse

\subsection{the discriminant}The discriminant of $K_{m,H}$ can be computed using Hasse's theorem..

\begin{Definition} (associated characters)

\end{Definition}
Theoretical result says that b
\fi

\end{document}
